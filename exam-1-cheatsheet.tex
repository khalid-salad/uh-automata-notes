\subsection{Exam 1}\label{subsec:exam-1-cheatsheet}
\subsubsection{Converting an NFA to a DFA}
Begin with the initial state. Apply the transitions and add any newly visited states. Stop when no new states can be visited. 

\subsubsection{Reducing a DFA}
Partition all states into accepting vs rejecting. Apply all transitions to a ``representative'' from a partition, then apply the transitions to the remaining members of the partition. If a member differs, move it to a new partition. 
\subsubsection{Converting a Regular Expression to an NFA}
\begin{enumerate}

\item[NFA for \(a\):]\begin{center}\begin{tabular}{r| c c r}
    & \(a\) & \(b\neq a\) & \\\bottomrule
    \(\to q_0\) & \(q_1\) & \(\emptyset\) & 0\\
          \(q_1\) & \(\emptyset\) & \(\emptyset\) & 1\\
\end{tabular}\end{center}

\item[NFA for \(\varepsilon\):]\begin{center}\begin{tabular}{r| c r}
    & \(c\in A\) & \\\bottomrule
    \(\to q_0\) & \(\emptyset\) & 1\\
\end{tabular}\end{center}

\item[NFA for \(\emptyset\):]\begin{center}\begin{tabular}{r| c r}
    & \(c\in A\) & \\\bottomrule
    \(\to q_0\) & \(\emptyset\) & 0\\
\end{tabular}\end{center}
\end{enumerate}

\begin{center}\textbf{Union}\end{center} 

Union initial states, copy remaining states from \(\alpha\) and \(\beta\). Final states are final states from \(\alpha\) and \(\beta\). 

\begin{center}\textbf{Concatenation}\end{center}

For final states of \(\alpha\), union the state from \(\alpha\) with the initial state from \(\beta\). Copy non-final states from \(\alpha\) and non-initial states from \(\beta\). Final states:

If the initial state is rejecting in \(\beta\), final states are final states from \(\beta\). 

If the initial state is accepting in \(\beta\), final states are final states from \(\alpha\) and \(\beta\), except the initial state in \(\beta\). 

\begin{center}\textbf{Kleene Closure}\end{center}

Union final states with the initial state. Copy non-final states. Final states are final states from \(\alpha\) and the initial state.

\subsubsection{Converting an NFA to a Regular Expression}

Set up system of equations. Solve for equation corresponding to initial state. Remember the lemma: if \[X=LX\cup M\] then \[X=L^*M\]