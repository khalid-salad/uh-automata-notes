\subsubsection{Extended Regular Expressions}\label{subsubsec:extended-regular-expressions"}
The languages we have discussed so far are \textbf{regular languages}. That is, 

\begin{itemize}
      \item Deterministic Finite Automaton
      \item Non-Deterministic Finite Automaton
      \item Regular Expression
      \item Solution of Languages Equations
\end{itemize}

are all regular languages. The following are \textbf{Closure Properties} of a regular language:

\begin{theorem}
      Let \(\mathcal{L_1}\) and \(\mathcal{L_2}\) be regular languages in some alphabet \(A\). Then
      \begin{enumerate}[1.]
            \item \(\mathcal{L}_1\cup\mathcal{L}_2\)
            \item \(\mathcal{L}_1\cdot \mathcal{L}_2\)
            \item \(\mathcal{L}_1^*\)
            \item \(\overline{\mathcal{L}_1}\)
      \end{enumerate}

      are all regular languages in \(A\).
\end{theorem}

\begin{proof}
      1, 2, and 3 follow from the definitions of regular expressions. For 4, consider a DFA \(\underset{\sim}{D}=(A, Q, \tau, q_0, \mathcal{F})\) and consider any word \(s\in A^*\). Further, let \(\underset{\sim}{D'}=(A, Q, \tau, q_0, Q-\mathcal{F})\). If \(w\in L(\underset{\sim}{D})\), then \(w\not\in L(\underset{\sim}{D'})\). On the other hand, if \(w\not\in L(\underset{\sim}{D})\), then \(w\in L(\underset{\sim}{D'})\). Then \(L(\underset{\sim}{D'})=\overline{L(\underset{\sim}{D})}\).
\end{proof}

This allows us to define the regular expression \(\overline{\alpha}\):

\begin{definition}
      Let \(\alpha \) be any regular expression in some alphabet \(A\). Then the regular expression \(\overline{\alpha}\) is defined by \[\overline{\alpha}=\overline{L(\alpha)}\] If a regular expression contains a complement, it is an \textbf{extended regular expression}.
\end{definition}

We can construct the DFA of the complement of a regular expression by finding the corresponding DFA and swapping the accepting and rejecting states. For example, consider the regular expression \(\overline{01^*}\cap\overline{10^*}\) over \( \{0, 1\} \). 

\[\overline{01^*}\cap\overline{10^*}=\overline{\overline{01^*}\cup\overline{10^*}}\]

Similarly, we consider the example \(\overline{{(\overline{01^*}0)}^*}\) over \( \{0,1,2\} \).

It should be noted that the above process of swapping accepting and rejecting states \textit{only works on a DFA}. Thus, if you wish to take the complement of an NFA, you must first convert it to a DFA\@. 

